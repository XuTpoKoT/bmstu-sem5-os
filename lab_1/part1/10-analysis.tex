\setcounter{page}{2}
%\section*{Цель работы}

%Знакомство со средством дизассемблирования Sourcer, получение дизассемблированного кода ядра операционной системы Windows на примере обработчика прерывания INT 8h в virtual mode – специальном режиме защищенного режима (32-разрядный режим работы), который эмулирует реальный режим работы  вычислительной системы на базе процессоров Intel.

%\section*{Задание}

%Используя Sourcer получить дизассемблированный код обработчика аппаратного прерывания от системного таймера INT 8h.

%На основе полученного кода составить алгоритм работы обработчика INT 8h.

\section*{Листинг кода}
 Листинги прерывания int 8h и процедуры sub\_1

\subsection*{Листинг INT8h} 
\begin{lstlisting}[style={asm}]
; Вызов процедуры sub_1 (запрет прерываний)
020A:0746  E8 0070				call	sub_1			; (07B9)

; Сохранение содержимого регистров ES, DS, AX, DX
020A:0749  06					push	es
020A:074A  1E					push	ds
020A:074B  50					push	ax
020A:074C  52					push	dx

; В регистр DS  записывается адрес 40h
; адрес области данных BIOS
020A:074D  B8 0040				mov	ax,40h
020A:0750  8E D8				mov	ds,ax

; В регистр ES записывается адрес 0h 
; (адрес таблицы векторов прерываний)
020A:0752  33 C0				xor	ax,ax			; Zero register
020A:0754  8E C0				mov	es,ax

; Инкремент младшей части счётчика СВ
020A:0756  FF 06 006C				inc	word ptr ds:timer_low	; (0040:006C=817h)

; Если младшая часть счетчика СВ == 0,
; то инкремент двух старших байтов СВ
; иначе переходим на loc_1
020A:075A  75 04				jnz	loc_1			; Jump if not zero

; Инкремент старшей части счётчика СВ
020A:075C  FF 06 006E				inc	word ptr ds:timer_hi	; (0040:006E=11h)

; Если два старших байта счетчика СВ == 24
; то сравниваем два младших байта счетчика СВ
; иначе декемент счетчика СВ до отключения моторчика дисковода
020A:0760			loc_1:						;  xref 020A:075A
020A:0760  83 3E 006E 18			cmp	word ptr ds:timer_hi,18h	; (0040:006E=11h)
020A:0765  75 15				jne	loc_2			; Jump if not equal

; Если два младших байта счетчика СВ == 176
; то обнуление счетчика СВ и установка флага прошедших суток
; иначе декемент счетчика времени до отключения моторчика дисковода
020A:0767  81 3E 006C 00B0			cmp	word ptr ds:timer_low,0B0h	; (0040:006C=817h)
020A:076D  75 0D				jne	loc_2			; Jump if not equal

; Обнуляем счетчик ( если прошел день )
020A:076F  A3 006E				mov	ds:timer_hi,ax		; (0040:006E=11h)
020A:0772  A3 006C				mov	ds:timer_low,ax		; (0040:006C=817h)

; В ячейку 0040:0070 записываем единицу 
; (Для фиксации о том , что новый день наступил )
020A:0775  C6 06 0070 01			mov	byte ptr ds:timer_rolled,1	; (0040:0070=0)
020A:077A  0C 08				or	al,8

; Декремент счетчика времени до отключения моторчика дисковода
020A:077C			loc_2:						;  xref 020A:0765, 076D
020A:077C  50					push	ax
020A:077D  FE 0E 0040				dec	byte ptr ds:dsk_motor_tmr	; (0040:0040=0CAh)

; Если значени этого счетчика == 0
; то установка флага отключения моторчика и посылка команды в порт на отключения моторчика
020A:0781  75 0B				jnz	loc_3			; Jump if not zero
020A:0783  80 26 003F F0			and	byte ptr ds:dsk_motor_stat,0F0h	; (0040:003F=0)
020A:0788  B0 0C				mov	al,0Ch
020A:078A  BA 03F2				mov	dx,3F2h
020A:078D  EE					out	dx,al			; port 3F2h, dsk0 contrl output


020A:078E			loc_3:						;  xref 020A:0781
020A:078E  58					pop	ax

; Проверка, установлен ли PF(parity flag) по адресу 0040:0314, 
; т.е. разрешен ли ответ на маскируемые прерывания
; (0100, установлен 2 бит, отвечает за флаг PF, флаг четности)
020A:078F  F7 06 0314 0004			test	word ptr ds:data_12e,4	; (0040:0314=3200h)

; Если вызов маскируемых прерываний разрешен, переход к вызову int 1Ch (в loc_4)
020A:0795  75 0C				jnz	loc_4			; Jump if not zero
020A:0797  9F					lahf				; Load ah from flags
020A:0798  86 E0				xchg	ah,al
020A:079A  50					push	ax

;  иначе, косвенный вызов 1Сh - как процедуры командой call и переход к loc_5
;  (1C * 4 = 70h )
020A:079B  26: FF 1E 0070			call	dword ptr es:data_4e	; (0000:0070=6ADh)
020A:07A0  EB 03				jmp	short loc_5		; (07A5)

; вызов пользовательского прерывания по таймеру	
020A:07A3			loc_4:						;  xref 020A:0795
020A:07A3  CD 1C				int	1Ch			; Timer break (call each 18.2ms)
; после инициализации системы вектор INT 1Ch указывает на команду IRET

; сброс контроллера прерываний
020A:07A5			loc_5:						;  xref 020A:07A0
020A:07A5  E8 0011				call	sub_1			; (07B9)
020A:07A8  B0 20				mov	al,20h			; ' '
020A:07AA  E6 20				out	20h,al			; port 20h, 8259-1 int command
;  al = 20h, end of interrupt

; восстановление значений регистров
020A:07AC  5A					pop	dx
020A:07AD  58					pop	ax
020A:07AE  1F					pop	ds
020A:07AF  07					pop	es

; прыжок в адрес 020A:064C
020A:07B0  E9 FE99				jmp	$-164h

; ---
020A:06AC  CF				iret	 ; Interrupt return

\end{lstlisting}

\clearpage

\subsection*{Листинг sub\_1} 
\begin{lstlisting}[style={asm}]	
				sub_1		proc	near
				
; Сохранение содержимого регистров DS, AX
020A:07B9  1E					push	ds
020A:07BA  50					push	ax

; В регистр DS  загружается адрес 0040:0000 начало области данных BIOS
020A:07BB  B8 0040				mov	ax,40h
020A:07BE  8E D8				mov	ds,ax

; Загрузка AH
020A:07C0  9F					lahf				; Load ah from flags

; Если флаг DF == 0 и старший бит IOPL == 0
; то сброс флага разрешения прерывания IF в 0040:0314
; иначе запрет маскируемых прерываний инструкцией CLI
020A:07C1  F7 06 0314 2400			test	word ptr ds:data_12e,2400h	; (0040:0314=3200h)
020A:07C7  75 0C				jnz	loc_7			; Jump if not zero

; Сброс флага IF 
020A:07C9  F0> 81 26 0314 FDFF	                           lock	and	word ptr ds:data_12e,0FDFFh	; (0040:0314=3200h)
020A:07D0			loc_6:						;  xref 020A:07D6

; Восстановление значений флагов
020A:07D0 9E					sahf				; Store ah into flags

; Восстановление значений регистров
020A:07D1  58					pop	ax
020A:07D2  1F					pop	ds
020A:07D3  EB 03				jmp	short loc_ret_8		; (07D8)

; Сброс IF, т. е. запрет прерываний с помощью команды cli
020A:07D5			loc_7:						;  xref 020A:07C7
020A:07D5  FA					cli				; Disable interrupts
020A:07D6  EB F8				jmp	short loc_6		; (07D0)

020A:07D8			loc_ret_8:					;  xref 020A:07D3

; Выход из программы
020A:07D8  C3					retn
				sub_1		endp

\end{lstlisting}

\clearpage

\section*{Схема алгоритма}

\img{220mm}{int8h_1.png}{Схема обработчика прерываний INT 8h}

\img{220mm}{int8h_2.png}{Схема обработчика прерываний INT 8h}

\img{220mm}{int8h_3.png}{Схема обработчика прерываний INT 8h}

\img{220mm}{sub.png}{Схема процедуры sub\_1}

\clearpage

%\section*{Вывод}

%Функции обработчика прерывания INT 8h в DOS:

%\begin{itemize}
%	\item Увеличивает текущее значение четырехбайтовой переменной, располагающейся в области данных BIOS по адресу 0000:046Ch. По этому адресу располагается счетчик тиков таймера. Если этот счетчик переполняется (после 24 часов с момента запуска таймера), в ячейку 0000:0470h заносится 1.
%	\item Контроль за работой двигателей моторчика дисковода. Если после последнего обращения к НГМД прошло более 2 секунд, обработчик прерывания выключает двигатель. Ячейка с адресом 0000:0440h содержит время, оставшееся до выключения двигателя. Это время постоянно уменьшается обработчиком прерывания таймера. Когда оно становится равно 0, двигатель НГМД отключается.
%	\item Вызов пользовательского прерывания 1Ch. Его стандартный обработчик состоит из одной команды IRET. Во время выполнения прерывания INT 1Ch все аппаратные прерывания запрещены.
%\end{itemize}