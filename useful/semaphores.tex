\chapter{Семафоры}
\section{Основная информация}
Семафор -- неотриц. защищённая переменная, на которой определены 2 операции:
P(S) - захват.
P(V) - освобождение.

Виды семафоров:
\begin{itemize}
	\item Бинарный
	\item Считающий ($xN0$) (контр. заполненность буфера)
	\item Множественный (наборы (массивы) считающих семафоров)
\end{itemize}

Какие-то переменные:
\begin{itemize}
	\item sc
	\item sf
	\item sb - число заполенных ячеек буфера
\end{itemize}

При использовании семафора мы используем системные вызовы -> переходим в режим ядра.
Переход в режим ядра -- плата за переход в режим активного ожидания.
Алгоритм булочная -- очередь типа фифо. В нём два процесса одновременно входят в дверь. В этом алг. есть акт. ожидание.
Процесс ждёт, пока все проц.  с большими идент. освободят...

Семафоры поддерживаются системой. Семантически они представляются в виде массивов -> в наборе семафоры обозначаются индексами(здесь это номер семафора в наборе, а не смещение). В адресном пр-ве ядра есть таблица семафоров. В ней каждый набор семафоров имеет дескриптор.

В дескрипторе есть:
\begin{itemize}
	\item идентификатор(его присв. процесс, создавший набор). Другие процессы могут получить этот ид. для доступа к семафору
	\item uid и ид. его группы . Эффективный uid которого совп. с uid создателя может ... и изменять его управляющие параметры.
	\item права доступа (для user, group, others)
	\item кол-во семафоров 
	\item время изм. семафоров последним процессом
	\item время после изменения упр. параметров
	\item указатель на массив семафоров.
\end{itemize}

О каждом семафоре известно:
\begin{itemize}
	\item значение семафора
	\item id процесса, который оперировал семафором в последний раз
	\item кол-во процессов, заблокированных на данном семафоре
\end{itemize}

Особенности семафоров:
\begin{itemize}
	\item У семафора нет хозяина -> освободить его может любой процесс(в отл. от ...)
\end{itemize}

В SystemV для каж. семафора определены сист. вызовы:
\begin{itemize}
	\item semget - получение семафора
	\item semctl - контроль
	\item semop - операция на семафоре
\end{itemize}
Одной неделимой операцией можно изменить несколько семафоров набора.

\begin{lstlisting}[language=c]
int semop(int semfd, struct sembuf *opPts, int len);
// len -- кол-во семафоров в наборе
struct sembuf
{
	ushost sem_num;
	short sem_op;
	short sem_flag;
}
\end{lstlisting}
	
В отличае от семафоров Дейкстры, у Unix определено 3 операции:
\begin{itemize}
	\item semop > 0 освобождение процессом захваченного ресурса(ыход из крит. секции). Если к семафору есть очередь процессов, после этой операции 1-й семафор в оч. сможет получить доступ к ресурсу.
	\item semop = 0 перевод процессса в сост. ожидания освоб. семафора и его ресурса без захвата семафора (только как инф.)
	\item захват. Если процесс не может вып. декремент, то он будет блокирован на данном сем., т.е. поставлен к нему в очередь.
\end{itemize}

На семафоре определены флаги (в sem\_flag):
\begin{itemize}
	\item IPC\_NOWAIT - информарует ядро о нежелании процесса переходить в сост. ожидания. Нужно, чтобы избежать блокирования всех процессов. В силу того, что kill нельзя перехватить, пр-с не сможет освободить захваченный семафор. Чтобы избежать этой ситуациии исп. флаг sem\_undo он указывает ядру на необходимость отслеживать изм. значения семафора в результате вызова sem\_op, чтобы сист. при завершении процесса смогла отменить сделанные изм. , чтобы процессы не были забл. невечно.
\end{itemize}
 
 \begin{lstlisting}
int sem_ctl(int semfd, int num, int end, union semun avg);

#include <sys/types.h>
#include <sys/ipe.h>
#include <sys/sem.h>

struct sembuf sbuf[2] = {{0, -1, SEM_UNDO| IPC_NOWAIT}, {1, 0, 1});
int main() {
	int pems = S_SIRWXV | S_IRWXG | S_IRWX0;
	int fd = semget(100, 2 IPC_CTREATE, | pems);
	if (fd == -1) {
		perror("semget");
		exit(1);
	}
	if (semop(fd, buf, 2) == -1) {
		perror("semget");
		return 0;
	}
}
\end{lstlisting}

Id набора 100 - ключ доступа к сегменту. Если такого небора нет, будет выполнен semop. .. 1-й семафор будет захвачен

\section{Средства разделяемой памяти.}
Семафор -- средство взяимоисключения. Каналы -- средство передачи инф.
Средства разделяемой памяти (сегменты) -- это буфер (не FIFO) в области данных. 

На этот участок памяти пр-с получает указатель. Здесь не определены ср-ва взаимоискл. (в отл. от pipe).
Разд. сегменты созданы таким образом, что они не требуют копирования данных из адр. пространства пр-са в пр-во ядра.
и наоборот за счёт того, что пр-с получает адрес разделяемой памяти. -> реализация монопольного достпупа лежит на программисте.
В ядре системе есть таблица разд. сегментов памяти. 

\begin{itemize}
	\item shmget
	\item shmctl 
	\item shmat
	\item shmdt  
\end{itemize}

\begin{lstlisting}
#include <string.h>
#include <sys/shm.h>

struct sembuf sbuf[2] = {{0, -1, SEM\_UNDO| IPC\_NOWAIT}, {1, 0, 1});
int main() {
	int pems = S_SIRWXU | S_IRWXG | S_IRWX0;
	int fd = shmget(100, 1024, IPC_CTREATE, | pems); // созд. сег. разд. памяти с ид. 100
	if (fd == -1) {
		perror("shmget");
		exit(1);
	}
	char *adr = (char*) shmat(fd, 0, 0);
	if (adr == (char*)-1) {
		perror("shmmat");
		exit(1);
	}
	strcpy(adr, "aaaa");
	if (shmdt(adr) == -1) {
		perror("shmmat");
		return 0;
	}
}
\end{lstlisting}
Создаёт сегмент разделяемой памяти с идентификатором 100. Если его удалось создать, он получает указатель. Если удалось получить адрес разделяемой памяти, он...
Информация, записанная в разделяемый сегмент остаётся. Её сможет прочитать любой другой пр-с по идентификатору.

Системные ограничения для разденяемых сегментов:
\begin{itemize}
	\item SHMMNI - максисальное возможное число разделяемых сегментов в ядре. При попытке создать больше, процесс перейдёт в состояние ожидания до удаления какого-либо сегмента;
	\item SHMMIN - максимальный размер разделяемого сегмента;
	\item SHMMAX - максимальный размер разделяемого сегмента.
\end{itemize}