\chapter{Мониторы}
Проблемы семафоров решают мониторы. Это средства более высокого уровня, чем примитивы ядра.

Примитив -- низкоуровневые средства, предоставляемые в распоряжение процессов.

NB! команда test\_and\_set
Команды lock -- unlock
		wait -- post
		wait -- signal
		
Идея монитора -- унификация взяимодействия процессов. Монитор может быть реализован в ОС или в библиотеке языка (например, Cuncurrent Pascal)

Монитор -- это набор процедур и данных (монитор защищает свои данные, т.к. к ним можно обратиться только через процедуры). Пр-с, вызвавший процедуру, наз. процессом, находящимся в мониторе. Процедура может одновременно использовать только один процесс. Остальные рпоцессы ставятся к очередь к монитору.

Обычно монитор оперирует переменными типа Conditional (событие) c помощью ф-й wait (блокировка) и signal (разблокировка).

Виды мониторов:
\begin{itemize}
	\item Простой монитор;
	\item Кольцевой монитор;
	\item Монитор читатели-писатели.
\end{itemize}

\section{Простой монитор}
Обеспечивает выделение ресурса произвольному числу процессов.
Когда пр-с захватывает ресурс, вызывается acquire. 
\begin{lstlisting}[label=lst:SimpleMonitor,caption=Код монитора]
resource : monitor
var
	busy : logical
	x    : conditional // event
	
procedure acquire
begin
	if busy then wait(x);
	busy = true; // process get access to resource
end

procedure release
begin
	busy = false;
	signal(x);
end

Begin
	busy = false
End.
\end{lstlisting}

\section{Кольцевой монитор}

\section{Монитор читатели-писатели}
Процессы писатели могут изменять данные, поэтому они должны работать  в режиме монопольного доступа к раздел. данным. Если писатель изм. анные, другие читатели/писатели не могут к ним обратиться.
Процессы-читатели могут только читать данные. При этом они могут читать параллельно.
Пример -- продажа билетов.

\begin{lstlisting}[label=lst:RWMonitor,caption=Код монитора читатели-писатели]
monitor : resource
var
	nr              : integer; // reader
	wrt             : logical; // writer
	c_read, c_write : condition;
	
procedure start_read
begin
	if wrt or turn(c_write) // active writer exists or there are waiting readers in queue
		then wait(c_read);
	inc(nr);
	signal(c_read); // other readers in queue become active
end;

procedure stop_read
begin
	dec(nr);
	if nr = 0 then // no readers => can write
		signal(c_write);
end;

procedure start_write
begin
	if nr > 0 or wrt
		then wait(c_write);
	wrt = true;
end;

procedure stop_read
begin
	wrt = false;
	if turn(c_read) // there are waiting readers in queue
		signal(c_read);
	else 
		signal(c_write)
end;

Begin
nr = 
End.
\end{lstlisting}

Читатель начинает с вызова srart\_write и заканчивавает вызовом stop\_write. 
Данный алгоритм предотвращает бесконечное откладывание.

\section{Алгоритм-булочная (Бейкери-алгоритм)}
Базируется на системе "Take a number".

\chapter{Очереди сообщений}
Было придумано ещё одно средство взаимодействия процессов, чтобы обеспечить разные потребности процессов. 

Посылка и приём собщений - самый общий способ взаимодействия. В сетях специально упакованные сообщ. наз пакетами. В ядре есть таблица очередей сообщений.

\begin{lstlisting}
#include <sys/msg.h>
struct msgid_ds;
\end{lstlisting}

\begin{itemize}
	\item msgget;
	\item msgcnt;
	\item msgset;
	\item msgrec; // recieve
\end{itemize}

При посылке сообщения оно копируется из адресного пр-ва процесса(пр-ва пользователя) в адресное пр-во ядра, при получении -- наоборот.
msg\_spot - указатель на текст сообщения. 

Когда пр-с помещает сообщение  в очередь, ядро созд. для него новую запись и помещает её в конец списка, соотв. сообщ. указанной очереди. В каждой такой записи указывается тип сообщения, число байт в сообщении и указатель на область данных ядра, в которой находится сообщение.

Ядро коп. ... После этого пр-с, отправивший сообщ, может завершитья. (пр-с не должен блокироваться в ожидании получения сооб. другим процессом)
Когда к-либо пр-с выбирает сообщ. из очереди, ядро коп. его в адр. пр-во процесса-получателя, после этого сообщ. удаляется из очереди.

Способы выбора соощения.
\begin{itemize}
	\item Взять самое старое сообщ. (из головы очереди), независимо от его типа.
	\item Если идентификатор сообщ. совпадает с идентификатором, который указал пр-с. Если сущ. несколько сообщ. с таким ид., берётся самое старое.
	\item Пр-с может взять сообщени, числ. значение типа которого есть менше или равное значению типа, кот указал пр-с. Если этому условию соответствеют несколько сообщений, берётся самое старое. (см. диаграмму 3 состояния блокировки пр-са при передачи сообщ.)
\end{itemize}

Пример.
\begin{lstlisting}
#ifndef MSGMAX
#define MSGMAX 1024
#endif

struct mbuf {
long msgtype;
char mtext[MSGMAX];
} mobj = {15, "aaa"};

int main() {
	int fd = msgget(100, IPC_CREATE | IPC_EXCL | 0642);
	if (fd == -1 || msgsnd(fd, &mobj, strlen(mobj.mtext), + 1, IPC_NOWAIT))
		perror("message");
	
	return 0;
}
\end{lstlisting}

\begin{itemize}
	\item ;
	\item ;
\end{itemize}
\chapter{Очереди сообщений}